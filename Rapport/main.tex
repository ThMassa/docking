\documentclass[12pt]{report}
\usepackage[french]{babel}

\usepackage[T1]{fontenc}
\usepackage{amstext}
\usepackage{amssymb}
\usepackage{graphicx}
\usepackage[utf8]{inputenc}
\usepackage{babel}
\usepackage{hyperref}



\usepackage[a4paper,top=3cm,bottom=2cm,left=2cm,right=2cm,marginparwidth=1.75cm]{geometry}  % pour configurer les marges


\makeatletter

\providecommand{\tabularnewline}{\\}

\makeatother

\begin{document}


\begin{titlepage}

\newcommand{\HRule}{\rule{\linewidth}{0.5mm}} % Defines a new command for the horizontal lines, change thickness here

\center % Center everything on the page
 
%----------------------------------------------------------------------------------------
%   HEADING SECTIONS
%----------------------------------------------------------------------------------------

\vspace{3cm}
\textsc{\LARGE Kévin REN,  Théo MASSA \\[0.5cm] Guillaume GARDE, Hugo HOFMANN } \\ [1.5cm]
\textsc{\Large UV 6.1 -- Projet au lac de Guerlédan}\\[1.5cm]

%----------------------------------------------------------------------------------------
%   TITLE SECTION
%----------------------------------------------------------------------------------------
\HRule \\[0.4cm]
{ \huge \bfseries \textsc{Docking\\[0.3cm]}}
\HRule \\[.5cm]

%----------------------------------------------------------------------------------------
%   DATE SECTION
%----------------------------------------------------------------------------------------

\vspace{1cm}
{\huge \today}\\[1cm] % Date, change the \today to a set date if you want to be precise

%----------------------------------------------------------------------------------------
%   LOGO SECTION
%----------------------------------------------------------------------------------------
% \raggedright
\vspace{2cm}
% \includegraphics[width = 5.5cm]{logo.pdf}\\[1cm] % Include a department/university logo - this will require the graphicx package
\includegraphics[width=6cm]{imgs/logo_ensta.jpg}
\includegraphics[width=6cm]{imgs/logo-lab-sticc2.png}
\includegraphics[width=6cm]{imgs/logo_ubs_transparent.png} 
%----------------------------------------------------------------------------------------

\vfill % Fill the rest of the page with whitespace

\end{titlepage}

\tableofcontents


\chapter{Introduction}

\section{Présentation du sujet}

\section{Matériel à disposition}
\subsection{Dock}
\subsection{Drone AUV}
%//TODO préciser qu'il n'est disponible qu'à Guerlédan
\subsection{Rover}

\chapter{Conception du dock}
\section{Mise en place d'une base RTK}

\section{Mise en place du dock}

\section{Communication avec le reste du système}

\chapter{Stratégie d'approche de docking}

\section{Modèle cinématique}

\section{Filtre de Kalman}

\section{Algorithme de Kévin}

\chapter{Architecture logicielle}

\chapter{Essais à Guerlédan}

\section{Première semaine}

\section{Deuxième semaine}

\chapter{Conclusion}
\section{Résultats}

\section{Perspectives}
%//TODO mentionner : + de tests avec le vrai dock, approche avec d'autres capteurs (LIDAR, caméra)
   



\bibliographystyle{plain}
\bibliography{mabiblio}

\end{document}
