\input ifpdf.sty

\ifpdf
\documentclass[pdftex,beamer]{beamer}
\else
\documentclass[xcolor=pst,dvips,beamer]{beamer}
\fi

\usetheme{EnstaB}
\setbeamercovered{transparent}
\useoutertheme{infolines}
\usepackage [frenchb]{babel}
\usepackage[utf8]{inputenc}
\usepackage{verbatim}
\usepackage{pgf,tikz}
\usepackage{amsmath,amssymb,euscript}
\usepackage{eurosym}
\usepackage{float}
\usepackage{mathtools}
\usepackage{algorithm2e}
\usepackage{subcaption}
% \usepackage{algorithmic}

\usenavigationsymbolstemplate{}

%//TODO
\hypersetup{pdftitle={IAMOOC},
            pdfsubject={Interval Analysis},
            pdfauthor={Olivier Reynet <olivier.reynet at ensta-bretagne.fr>},
            pdfkeywords={Interval Analysis, Set Inversion, Contractors},
            %pdfpagemode={FullScreen}
}

\usepackage{multicol}
\unitlength 1cm
% ensembles
\RequirePackage{dsfont}
\def\nbR{\mathds{R}}
\def\nbN{\mathds{N}}
\def\nbQ{\mathds{Q}}
\def\nbZ{\mathds{Z}}


\theoremstyle{definition}
\newtheorem{defi}{Définition}
\theoremstyle{example}
\newtheorem{remarque}{Remarque}
\newtheorem{remarques}{Remarques}
\newtheorem{exemple}{Exemple}
\theoremstyle{plain}
\newtheorem{theoreme}{Théorème}
\newtheorem{propriete}{Propriétés}
\newtheorem{lemme}{Lemme}
\newtheorem{corollaire}{Corollaire}

\title[RRT*]{Constrained Path planning with RRT*}
\author[H. Hofmann]{Hugo Hofmann}
\date{\today}
\institute{ENSTA Bretagne}

\newcommand{\ct}{\cos(\theta)}
\newcommand{\st}{\sin(\theta)}
\usepackage{amssymb} % for \mathbb


\graphicspath{{../RRT-Path-Planning-GUI/imgs/},{../Rapport/}}


\begin{document}

\frame{\titlepage}

\section*{Introduction}

\begin{frame}
  \frametitle{Introduction - Path Planning}
  % \vspace{-2cm}

  Planning a path within a 2D or 3D space is fundamental in robotics.

  A few major algorithms :

  \begin{itemize}
    \item Djikstra
    \item A*
    \item RRT
  \end{itemize}


  \includegraphics[width=0.2\textwidth]{robot_arm.png}
  \includegraphics[width=0.2\textwidth]{bezier_curve_map_rviz.png}
  \includegraphics[width=0.55\textwidth]{autonomous_car.jpeg}



  
\end{frame}

\begin{frame}
  \frametitle{Introduction - Path planning}

  \begin{alertblock}{Limitations}
    Issues generally encountered:
    \begin{itemize}
      \item No pre-existing graph
      \item Convergence towards an optimal solution
      \item Account for complex path constraints
    \end{itemize}
  \end{alertblock}

  \begin{exampleblock}{Proposition}
    A variant of the RRT algorithm, \textit{RRT*}
  \end{exampleblock}

\end{frame}

\begin{frame}{Table of contents}
  \tableofcontents
  % possibilité d'ajouter l'option [pausesections]
\end{frame}

\section{RRT* algorithm}


\begin{frame}
  \frametitle{RRT* algorithm}

  \begin{algorithm}[H]
    \caption{RRT* Algorithm}
    \KwResult{RRT* tree T}
    
    Initialization\;
    
    \While{Not converged}{
        Sample random configuration $q_{\text{rand}}$\;
        Find nearest node $q_{\text{near}}$ to $q_{\text{rand}}$ in $T$\;
        Steer towards $q_{\text{rand}}$ from $q_{\text{near}}$ to get $q_{\text{new}}$\;
        \If{$q_{\text{new}}$ is collision-free}{
            Find nearby nodes $N_{q_{\text{new}}}$ in $T$\;
            Choose parent $q_{\text{min}}$ from $N_{q_{\text{new}}}$ minimizing cost\;
            Rewire $T$ to $q_{\text{new}}$ if it improves cost-to-come\;
            Add $q_{\text{new}}$ to $T$ with $q_{\text{min}}$ as parent\;
        }
    }
    
    \caption{RRT* Algorithm}
\end{algorithm}

\end{frame}

\section{Kinematic constraints}

\begin{frame}
  \tableofcontents[currentsection]
\end{frame}

\begin{frame}{Kinematic model}
  \centering
  \includegraphics[width=0.5\textwidth]{kinematic_model.png}

  \begin{equation}
    \begin{bmatrix}
      \dot{x}\\\dot{y}\\\dot{v}\\\dot{\theta}
    \end{bmatrix}
    = \begin{bmatrix}
      v \ct \\ v \st \\
      a \\ \frac{v \tan(\delta)}{L}
      \end{bmatrix}
    = f(X, u) \quad \text{with } u=[a,\delta]^T
  \end{equation}
\end{frame}

\begin{frame}{Control strategy}
  \begin{minipage}{0.5\textwidth}
    \begin{figure}[H]
      \begin{equation*}
        a_c = \frac{v^2}{R} = \frac{v^2\sin(\delta)}{L}
      \end{equation*}
      \begin{equation*}
        v_R = \min(v_{n+1}, \sqrt{\frac{a_{c_{max}} L}{\sin(\delta)}})
      \end{equation*}
    \end{figure}
    \end{minipage}%
    \begin{minipage}{0.5\textwidth}
    \centering
    \includegraphics[width=0.7\textwidth]{controlV2} % Replace example-image with your image file
    \end{minipage}
 

  \begin{block}{Commands}
    
    \begin{equation}
      \begin{cases}
        a = {clamp}_{[-a_{m},a_{m}]}(K_p(v_R-v)) \\
        \delta = {clamp}_{[-\delta_{m},\delta_{m}]}(K_{str} \arctan(K_{ct} e_{ct}) + K_{str} e_{\theta}) 
      \end{cases}
    \end{equation}
    
  \end{block}
\end{frame}

\section{Results}
\subsection{Without kinematic constraints}

\begin{frame}
  \tableofcontents[currentsection,currentsubsection]
\end{frame}


\begin{frame}{First results}
  \begin{figure}[H]
    \begin{subfigure}{0.49\textwidth}
      \centering
      \includegraphics[width=0.9\textwidth]{urban_map_path.png}
      \caption{Path found on an urban map}
    \end{subfigure}
    \begin{subfigure}{0.49\textwidth}
      \centering
      \includegraphics[width=0.9\textwidth]{sigma_maze_tree.png}
      \caption{Tree graph (green) and path (red) in an hexagonal maze}
    \end{subfigure}
    \caption{Shortest paths using euclidean distance and no kinematic constraints}
  \end{figure}
\end{frame}

\begin{frame}{First results}
  \begin{figure}[H]
    \centering
    \includegraphics[width=0.9\textwidth]{silverstone_shortest_path.png}
    \caption{Shortest (but obviously not fastest) trajectory on the Silverstone racetrack}
  \end{figure}
\end{frame}

\subsection{With kinematic constraints}

\begin{frame}
  \tableofcontents[currentsection,currentsubsection]
\end{frame}

\begin{frame}{Choice of the $C$ \textit{Cost} function}
  \begin{exampleblock}{Different choices for $C(x,u)$}
    \begin{enumerate}
      \item Path length : \quad $C(x,u) = \int_{0}^{t_{end}} |v(t)| {dt}$
      \item Total time : \quad $C(x,u) = t_{end}$
      \item Energy consumption, tire fatigue, ...
    \end{enumerate}
    
  \end{exampleblock}
\end{frame}

\begin{frame}{Finding the fastest path}
  \begin{figure}[H]
    \centering
    \begin{subfigure}{0.49\textwidth}
      \centering
      \includegraphics[width=0.95\textwidth]{nice_trajectory_cropped.png}
      \caption{Simulation of the fastest path with 2114 nodes}
      \label{fig:corner_constraints}
    \end{subfigure}
    \begin{subfigure}{0.49\textwidth}
      \includegraphics[width=0.95\textwidth]{silverstone_race_line.png}
      \caption{Reference race line on the Silverstone racetrack}
      \label{fig:corner_reference}
    \end{subfigure}
    \caption{Attempting to find the fastest path on the Silverstone racetrack}
  \end{figure}
\end{frame}

\section{Conclusion and perspectives}

\begin{frame}{Conclusion}
  \begin{minipage}{0.49\textwidth}
    
    \begin{alertblock}{Limitations}
      \begin{itemize}
        \item Implementation issues
        \item Optimization (ex: better node sampling)
      \end{itemize}
    \end{alertblock}
    
    \begin{exampleblock}{Results}
      \begin{itemize}
        \item Rather satisfactory results
        \item Many applications (ex: parking manoeuvers)
      \end{itemize}
    \end{exampleblock}

  \end{minipage}
  \begin{minipage}{0.5\textwidth}
    \hfill
    \begin{figure}[H]
      \centering
      \includegraphics[width=0.9\textwidth]{node_generation.png}
      \caption{Proposition of a dynamic generation of nodes}
      \label{fig:node_generation}
    \end{figure}
  \end{minipage}
\end{frame}

\begin{frame}[allowframebreaks]
  \frametitle{References}
  % Include all entries from the bibliography file
  \nocite{*}
  % Specify your bibliography file
  \bibliography{../Rapport/References}
  \bibliographystyle{plain}
\end{frame}

\begin{frame}{Conclusion}
  \begin{center}
    \Huge \textbf{End of the presentation}
  \end{center}
\end{frame}


\end{document}
